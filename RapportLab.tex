\documentclass[10pt,a4paper]{report}
\usepackage[utf8]{inputenc}
\usepackage{amsmath}
\usepackage{amsfonts}
\usepackage{amssymb}
\usepackage{graphicx}
\usepackage{hyperref}
\usepackage{tabto}
\usepackage{gensymb}
\usepackage{booktabs,caption,fixltx2e}
\usepackage[flushleft]{threeparttable}
\usepackage{tocloft}
\usepackage[margin=1in]{geometry}
\author{Finn Matras, Jakob Løver}
\title{{\LARGE TFY4115}\\{\large Lab 1, Friksjon på skråplan}}
\begin{document}
\renewcommand{\contentsname}{Innhold}
\renewcommand{\cftchapleader}{\cftdotfill{\cftdotsep}}
\renewcommand{\cftpartleader}{\cftdotfill{\cftdotsep}}

\maketitle
\tableofcontents
%\chapter*{Innhold}
%\begin{itemize}
%\item Sammendrag \tab{1}			
%\item Innledning \tab{2}
%\item Teori      \tab{3}
%\item Eksperimentelt \tab{4}
%\item Resultater og diskusjon \tab{5}
%\item Konklusjon \tab{6}
%\end{itemize}

\chapter*{Sammendrag}
\addcontentsline{toc}{chapter}{Sammendrag}
Formålet med laben var å regne seg frem til den horisontale vinkelen til et skråplan som gjør at et system bestående av to masser sklir med konstant hastighet.\\
\\For å finne friksjonskoeffisienten ble videoanalyse-programvaren Tracker og et høyhastighetskamera brukt for å analysere bevegelsen til klossene. I Tracker beregnes akselerasjonen til klossen, og ved hjelp av Newtons II lov kan friksjonskoeffisienten til underlagene kalkuleres. Målingene ble gjenntatt flere ganger for å verifisere og ta høyde for usikkerhet.\\
\\Kalkuleringene som ble foretatt viser til en friksjonskoeffisient mellom nylon (Figur: 1) og skråplanet i tre på 0.232, og 0.270 mellom plast og tre (Figur: 2).


%\chapter*{Innledning}
{\let\clearpage\relax\chapter*{Innledning}}
\addcontentsline{toc}{chapter}{Innledning}
\section*{Teori}
Basert på Newtons II lov som i likning (2): 
\begin{equation}
\sum{F} = m \cdot a,
\end{equation} der $F$ er kraften, $m$ er massen, og $a$ er akselerasjonen, så kan man regne ut kreftene som er tilstede på en masse. Hvis $a$ er 0, så får man likning (2) for klossene i figur 3. 
\begin{equation}
F = \mu \cdot F_n.
\end{equation}
I likningen over er $F$ kraften som trekker nedover, $mu$ er friksjonskraften, og $F_n$ er normalkraften til klossen på planet.
Ved å bruke likning (1) og (2), får man likning (3) og (4)
\begin{equation}
\mu = \tan(\theta)-a/(\cos(\theta)\cdot g).
\end{equation}
Der $\theta$ er vinkelen mellom skråplanet og et horisontalt plan, $\mu$ er friksjonskoeffisienten, $a$ er akselerasjonen og $g$ er gravitasjonsakselerasjonen. Ved hjelp av denne liknignen kan man regne ut friksjonskoeffisienten for de forskjellige akselerasjonsmålingene. Ved å regne på likning (2) og (3) og løser for vinkelen $\theta$ så får man
\begin{equation}
\theta = \arctan(\frac{\mu_1 \cdot m_1+\mu_2 \cdot m_2}{m_1+m_2})
\end{equation}
der $\mu_1$ er friksjonskoeffisienten til kloss 1, og $m_1$ er massen til samme kloss. $\mu_2$ og $m_2$ tilhører kloss 2. Slik kan man regne seg fram til vinkelen, $\theta$, som gjør at klossene med masser $m_1$ og $m_2$ sklir med konstant fart ned langs skråplanet for et massesystem på et gitt underlag med friksjonskoeffisienter, $\mu_1$ og $\mu_2$.\\
\\I dette forsøket var formålet å bli kjent med hvordan man utfører eksperimentelle forsøk i fysikken, og å bli kjent med å skrive lab-journal og rapport. Vi har lært mye om usikkerhet og hvordan feil i det tekniske utstyret kan føre til følgefeil.

\chapter*{Metoder og eksperiment}
\addcontentsline{toc}{chapter}{Metoder og eksperiment}
\section*{Metode}
\addcontentsline{toc}{section}{Metode}
\subsection*{Forberedelse}
\addcontentsline{toc}{subsection}{Forberedelse}
Et skjematisk oppsett av eksperimentet er vist i Figure \ref{oppsett}. Treplanken ble festet til et stativ med en klype. Det ble brukt vater for å påse at skråplanets vinkel blir målt normalt til tyngekraften. Foran oppsettet ligger det en meterstokk, noe som gjør det enkelt å kalibrere størrelsesforhold i programvaren Tracker. Høyhastighetskameraet er montert normalt på oppsettet, slik at vinkling på kameraet ikke påvirker størrelsene som måler. To forskjellige typer underlag ble brukt, som vist i Figure \ref{fig:1} og Figure \ref{fig:2}. De forskjellige underlagene ble festet på klosser av aluminium og messing ved hjelp av lærertyggis.\\
\\Som foreberedelse til laben ble likningene til laben utledet, beskrevet under seksjonen "Teori." Repetitive kalkulasjoner ble utført ved hjelp av et digital regneark. Innstillingene på høyhastighetskameraet ble tilpasset lysforholdene på laben. Dette resulterte i at vi brukte 100fps med automatisk hvitbalanse.\\
\\En rekke eksperimenter ble fullført for å verifisere at utstyret fungerte. Ved hjelp av en vekt ble de to massenes vekt målt tre ganger. Gjennomsnittet av de tre målingene verifiserte at målingene var representative tall for vekten til massene. En av vektene ble sluppet foran en hvit vegg mens kameraet filmet. Dette filmklippet ble importert i Tracker. Ved hjelp av regresjonsfunksjonen i programmet ble integriteten til utstyret verifisert da denne gav 9.81 m/s$^2$ som akselerasjon.

\begin{figure}
    \centerline{\includegraphics[scale=0.05]{BlaaPlast}}
    \caption{Blå plast, friksjonskoeffisient: 0.270.}
    \label{fig:1}
\end{figure}
\begin{figure}
    \centerline{\includegraphics[scale=0.05]{GultStoff}}
    \caption{Gul nylon, friksjonskoeffisient: 0.232.}
    \label{fig:2}
\end{figure}

\subsection*{Forsøk}
\addcontentsline{toc}{subsection}{Forsøk}
Forsøket ble gjennomført ved å først stille inn skråplanet til en vilkårlig vinkel. Med en linjal målte vi lengden og høyden av skråplanet, og brukte den inverse tangensen for å kalkulere vinkelen til skråplanet. Når kameraet rullet, slapp vi messingklossen 4 ganger nedover skråplanet. Dette importerte vi i Tracker for å beregne akselerasjon. Ved hjelp av regnearket vårt og formlene definert ovenfor, regnet vi oss fram til friksjonskoeffesienten for klossen. Dette gjentok vi med aluminiumsklossen, og deretter med både nylonbiten og plastbiten som underlag. \\
\\Friksjonskoeffesientene for underlagene vi valgte brukte vi til å beregne oss frem til den optimale vinkelen hvor begge klossene ville skli med konstant hastighet, som beskrevet i "Teori" seksjonen. Klossene ble bundet sammen med en tråd, og sluppet ned skråplanet. Dette forsøket ble også tatt opp ved hjelp av Tracker hvor vi målte en neglisjerbar akselerasjon, noe som bekreftet beregningene våre om optimal vinkel.\\
\\Til slutt ble vi bedt om å kalkulere optimal vinkel da det ble lagt til 10g og 50g på hver av klossene. Da vi allerede hadde kalkulert friksjonskoeffesientene til underlagene i de forrige eksperimentene, kunne vi enkelt kalkulere vinkelen med de samme formlene. Denne gangen målte vi også en neglisjerbar akselerasjon.


\begin{figure}
\centerline{\includegraphics[scale=0.5]{oppsett}}
\caption{Illustrasjon av oppsettet.}
\label{oppsett}
\end{figure}


{\let\clearpage\relax\chapter*{Resultater og diskusjon}}
\addcontentsline{toc}{chapter}{Resultater og diskusjon}
\section*{Resultater}
\addcontentsline{toc}{section}{Resultater}
Nedenfor er resultatene vi fikk ved å bruke Tracker til å finne akselerasjonen til klossene, ved hjelp av en regresjon av posisjonsgraf av bevegelsen til klossene. Vinkelen ble også funnet i Tracker, ved å se på vinkel mellom skråplanet og vateret. Merk at i hver tabell var vinkelen, materiale og klosstype lik.
\begin{center}
  \begin{tabular}{| c | c | c | c | c | c |}
    \hline
    Måling & Klosstype & Materiale & Akselerasjon $[m/s^2]$ & Vinkel [$^{\circ}$] & Friksjonskoeffisient | \\ \hline
    1 & Messing & Nylon & 0.248 & 17.800 & 0.268 $\pm 0.026$ \\ \hline
    2 & Messing & Nylon & 0.412 & 16.900 & 0.216 \\ \hline
    3 & Messing & Nylon & 0.370 & 16.800 & 0.223 \\ \hline
    4 & Messing & Nylon & 0.366 & 16.700 & 0.222 \\ \hline
  \end{tabular}
 \begin{tablenotes}
 	\small
 	\item Tabell 1, viser akselerasjon og friksjonskoeffisient for messingkloss på nylonunderlag. Vinkelen for alle forsøkene var lik, men pga. forskjellig oppsett i Tracker, så er det også en usikkerhet i denne.
 	\end{tablenotes}
\end{center}


\begin{center}
  \begin{tabular}{| c | c | c | c | c | c |}
    \hline
    Måling & Klosstype & Materiale & Akselerasjon $[m/s^2]$ & Vinkel [$^{\circ}$] & Friksjonskoeffisient | \\ \hline
    5 & Aluminium & Nylon & 0.339 & 17.200 & 0.237 \\ \hline
    6 & Aluminium & Nylon & 0.299 & 16.800 & 0.238 \\ \hline
    7 & Aluminium & Nylon & 0.294 & 17.800 & 0.258 \\ \hline
    8 & Aluminium & Nylon & 0.306 & 17.500 & 0.250 \\ \hline
  \end{tabular}
   \begin{tablenotes}
 	\small
 	\item Tabell 2, viser akselerasjon og friksjonskoeffisient for aluminiumskloss på nylonunderlag.
 	\end{tablenotes}
\end{center}


\begin{center}
  \begin{tabular}{| c | c | c | c | c | c |}
    \hline
    Måling & Klosstype & Materiale & Akselerasjon $[m/s^2]$ & Vinkel [$^{\circ}$] & Friksjonskoeffisient | \\ \hline
    9 & Messing & Plast & 0.337 & 18.200 & 0.256 \\ \hline
    10 & Messing & Plast& 0.196 & 18.700 & 0.296 \\ \hline
    11 & Messing & Plast& 0.296 & 17.800 & 0.258 \\ \hline
    12 & Messing & Plast& 0.189 & 18.200 & 0.288 \\ \hline
  \end{tabular}
     \begin{tablenotes}
 	\small
 	\item Tabell 3, viser akselerasjon og friksjonskoeffisient for messingkloss på plastunderlag.
 	\end{tablenotes}
\end{center}

For å beregne farten til aluminiumsklossen med plastunderlag, så måtte vi endre vinkelen for at klossen skulle skli.
\begin{center}
  \begin{tabular}{| c | c | c | c | c | c |}
    \hline
    Måling & Klosstype & Materiale & Akselerasjon $[m/s^2]$ & Vinkel [$^{\circ}$] & Friksjonskoeffisient | \\ \hline
    13 & Aluminium & Plast & 0.431 & 25.000 & 0.369 \\ \hline
    14 & Aluminium & Plast & 0.418 & 24.900 & 0.370 \\ \hline
    15 & Aluminium & Plast & 0.572 & 26.100 & 0.360 \\ \hline
    16 & Aluminium & Plast & 0.766 & 25.300 & 0.300 \\ \hline
  \end{tabular}
     \begin{tablenotes}
 	\small
 	\item Tabell 4, viser akselerasjon og friksjonskoeffisient for aluminiumskloss på plastunderlag.
 	\end{tablenotes}
\end{center}

\begin{center}
  \begin{tabular}{| c | c | c | c | c |}
    \hline
    Klosstype & Materiale & Akselerasjon $\pm \delta A$ $[m/s^2]$ & Vinkel $\pm \delta V$ [$^{\circ}$] & Friksjonskoeffisient | \\ \hline
    Messing & Nylon & 0.349 $\pm 0.082$ & 17.100 $\pm 0.550$ & 0.232 $\pm 0.026$ \\ \hline
    Aluminium & Nylon & 0.310 $\pm 0.023$ & 17.300 $\pm 0.500$ & 0.246 $\pm 0.011$\\ \hline
    Messing & Plast & 0.255 $\pm 0.074$ & 18.200 $\pm 0.450$ & 0.270 $\pm 0.020$\\ \hline
    Aluminium & Plast & 0.547 $\pm 0.174$ & 25.300 $\pm 0.600$ & 0.350 $\pm 0.035$\\ \hline
  \end{tabular}
 \begin{tablenotes}
 	\small
 	\item Tabell 5, viser snitt for akselerasjon, vinkel og friksjonskoeffisient for de forskjellige materialene og klossene.
 	\end{tablenotes}
\end{center}

Ved å sette verdiene for snitt, med usikkerhet inn i linkning (1), og få følgende verdier for vinkelen, $\theta$, for et system bestående av to klosser som i Figure 3:

\begin{center}
  \begin{tabular}{| c | c | c |}
    \hline
    Stoff kloss 1 & Stoff kloss 2 & $\overline{\theta} \pm \delta\theta$ \\ \hline
    Nylon & Nylon & 13.6 $\pm$ 0.8 \\ \hline
    Plast & Plast & 16.4 $\pm$ 1.2 \\ \hline
    Nylon & Plast & 17.7 $\pm$ 1.7 \\ \hline
    Plast & Nylon & 14.7 $\pm$ 1.5  \\ \hline
  \end{tabular}
     \begin{tablenotes}
 	\small
 	\item Tabell 6, viser vinkelen, $\theta$ som teoretisk sett skal gi konstant hastighet ned skråplanet, med usikkerhet.
 	\end{tablenotes}
\end{center}
Testing av utregnede verdier på et massesystem som vist i Figure 3 ga oss følgende akselerasjoner:

\begin{center}
  \begin{tabular}{| c | c | c | c | c  | c |}
    \hline
    Test nr. & Stoff & Vekt kloss 1 [kg] & Vekt kloss 2 [kg] & Vinkel $\theta$ & Akselerasjon $[m/s^2]$ | \\ \hline
    1 & Nylon & 0.022 & 0.0687 & 12.8 & 0,12 \\ \hline
    2 & Nylon & 0.0242 & 0.0687 & 12.4 & 0,02 \\ \hline
    3 & Nylon & 0.0343 & 0.1197 & 12.9 & 0,04 \\ \hline
    4 & Plast & 0.0343 & 0.1197 & 16.0 & lav, ingen korrekt måling  \\ \hline
  \end{tabular}
     \begin{tablenotes}
 	\small
 	\item Tabell 7, viser resultatene fra målingene som ble gjennomført på klosser med gitt vekt og utregnet friksjonskoeffisient og vinkel for konstant akselerasjon, altså ingen akselerasjon.
 	\end{tablenotes}
\end{center}

\section*{Diskusjon og feilkilder}
\addcontentsline{toc}{section}{Diskusjon og feilkilder}
Som en ser i resultatene ovenfor, så ga alle de teoretisk utregnede verdiene en vinkel som var for stor, noe som førte til at klossene akselererte nedover skråplanet. En feilkilde som ble observert mens testene ble gjennomført var at klossene beveget seg litt hakkete nedover skråplanet. Dette er mest sannsynlig forårsaket ved at skråplanet ikke har helt homogen overflate, noe som gjør at klossene og friksjonskoeffisientene endrer seg underveis. Klossene ble ikke alltid sluppet ned på akuratt samme sted, da skråplanet var forholdsvis bredt, og overflaten kan også være forskjellig i bredden, noe som kan ha forårsaket feil i målingne. Ut ifra likning (3) ser man også at vinkelen, $\theta$ har mye å si, og dette ble også observert ved de initielle målingene. Vateret ble brukt til å finne nøyaktig horisontal posisjon var svært viktig, men her kan det fortsatt være en del usikkerhet, med tanke på hvordan det ble overført til Tracker. \\
\\I målingene i Tabell 5 ser vi også at friksjonskoeffisienten er en del større, enn den er i Tabell 3. Her er både massen og vinkelen endret. Vel å merke her er at det er forskjell på friksjon avhengig av hastighet mellom de to underlagene. Det vil si at det for plasten var en annen friksjonskoeffisient ved høyere hastigheter enn ved lavere hastigheter. Dermed kunne vært lurt å prøve seg fram til den vinkelen der det er konstant hastighet eksperimentelt, og så analysere akselerasjonen ved den funnede vinkelen for å finne friksjonskoeffisienten slik den er i nærheten av den hastigheten som den kommer til å bli utsatt for.\\
\\En annen ting som ble observert i mvålingene som ble gjennomført helt i starten av en kloss i fritt fall, var at selv om klossen falt fritt, så stemte ikke alltid akselerasjonsverdiene som vi fikk ved regresjonen i Tracker. Disse verdiene lå i nærheten av det en skulle forvente, men at verdiene hadde en usikkerhet på rundt $\pm$ 20$\%$. Derfor ble det tatt totalt 8 målinger av hver stofftype, men for enda mer presise resultater kunne det blitt gjort en rekke flere målinger.

{\let\clearpage\relax\chapter*{Konklusjon}}
\addcontentsline{toc}{chapter}{Konklusjon}
Hypotesen vår var at vi ville klare å kalkulere optimal vinkel for konstant akselerasjon, noe vi kan konkludere med at vi klarte. Det var en del feilkilder som gjorde at resultatene våre ikke ble slik vi håpet på som beskrevet i seksjonen over. Vi opplevde ofte at vi fikk en for høy vinkel da vi kalkulerte vinkelen, noe vi så var et gjengående moment hos de andre gruppene. Etter å ha sammenliknet friksjonskoeffisientene til underlagene med nettet [1], kan vi også konkludere at vi klarte dette. Utfordringen hvor vi ble bedt om å finne vinkel til to forskjellige masser med eget valg av friksjonsunderlag klarte vi også å bestå, dog med samme problem hvor vi måtte senke planet med et par graders avvik fra beregningene våre. Videre har vi også lært oss hvordan vi bør føre journal for at vi eller andre i etterkant selv skal kunne gjennomføre forsøket på nytt.

\chapter*{Referanser}
[1] \href{url}{https://en.wikipedia.org/wiki/Friction}
Nedlastingsdato: 26. oktober 2016

\end{document}

