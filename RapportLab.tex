\documentclass[10pt,a4paper]{report}
\usepackage[utf8]{inputenc}
\usepackage{amsmath}
\usepackage{amsfonts}
\usepackage{amssymb}
\usepackage{graphicx}
\usepackage{hyperref}
\usepackage{tabto}
\usepackage{gensymb}
\usepackage{booktabs,caption,fixltx2e}
\usepackage[flushleft]{threeparttable}
\usepackage{tocloft}
\usepackage[margin=0.8in]{geometry}
\author{Finn Matras, Jakob Løver}
\title{{\LARGE TFY4115}\\{\large Friksjon på skråplan}}
\begin{document}
\renewcommand{\contentsname}{Innhold}
\renewcommand{\cftchapleader}{\cftdotfill{\cftdotsep}}
\renewcommand{\cftpartleader}{\cftdotfill{\cftdotsep}}

\maketitle
\tableofcontents
%\chapter*{Innhold}
%\begin{itemize}
%\item Sammendrag \tab{1}			
%\item Innledning \tab{2}
%\item Teori      \tab{3}
%\item Eksperimentelt \tab{4}
%\item Resultater og diskusjon \tab{5}
%\item Konklusjon \tab{6}
%\end{itemize}

\chapter*{Sammendrag}
\addcontentsline{toc}{chapter}{Sammendrag}
Formålet med laben var å regne seg frem til den horisontale vinkelen til et skråplan som gjør at et system bestående av to masser sklir med konstant hastighet. Friksjonskoeffisienten estimeres ved hjelp av Newtons II lov og videoanalyse-programmvaren Tracker. Målingene ble gjenntatt flere ganger for å verifisere og ta høyde for usikkerhet.\\
\\Kalkuleringene som ble foretatt viser til en friksjonskoeffisient mellom nylon og skråplanet i tre på 0.232 $\pm$ $0.026$, og 0.270 $\pm$ $0.020$ mellom plast og tre. Akselerasjonen som ble målt med vinkel $\theta = 12.4 ^{\circ}$ var 0,02 $m/s^2$.

{\let\clearpage\relax\chapter*{Innledning}}
\addcontentsline{toc}{chapter}{Innledning}
I dette forsøket var formålet å bli kjent med hvordan en utfører eksperimentelle forsøk i fysikken, og å bli kjent med å skrive lab-journal og rapport. Den eksperimentelle delen ble utført ved hjelp av regresjoner av bevegelsen til en kloss som ble analysert i Tracker, ved så å regne seg fram til friksjonskoeffisienten. Deretter ble det satt opp likninger for et massesystem og vinkelen som fører til konstant hastighet regnet ut.


{\let\clearpage\relax\chapter*{Teori}}
\addcontentsline{toc}{chapter}{Teori}
Basert på Newtons II lov har vi at 
\begin{equation}
\sum{F} = ma,
\end{equation} der $F$ er kraften, $m$ er massen, og $a$ er akselerasjonen, så kan kreftene som virker på en masse regnes ut. Likningen,
\begin{equation}
F_r = \mu F_n,
\end{equation}
der $F_r$ er friksjonskraften, $\mu$ er friksjonskoeffisienten, og $F_n$ er normalkraften til klossen på planet, gir et forhold mellom friksjonskraften og friksjonskoeffisienten. Ved å sette opp likning (1) for Figure 1 fås likning (3),
\begin{equation}
F_x - F_r = ma.
\end{equation}
 Ved hjelp av likning (2) og (3), kan likningen 
\begin{equation}
\mu = \tan(\theta)-a/(\cos(\theta)g),
\end{equation}


avledes, der $\theta$ er vinkelen mellom skråplanet og et horisontalt plan, $g$ er gravitasjonsakselerasjonen. \\
For å regne seg fram til vinkelen, $\theta$, mellom horisontalplanet og skråplanet, som gjør at massesystemet sklir med konstant fart ned langs skråplanet på et gitt underlag settes likning (1) opp for et system bestående av to masser, som vist i Figure 1. Det antas her at de to massene er bunnet sammen med en masseløs snor, som forblir stram og at snorkreftene er like store.
\begin{equation}
F_{xm1} - F_{rm1} + F_{xm2} - F_{rm2} = ma,
\end{equation}
løses for vinkelen, $\theta$, der $a = 0$, 
\begin{equation}
\theta = \arctan(\frac{\mu_1m_1+\mu_2m_2}{m_1+m_2}),
\end{equation}
der $m_1$ og $m_2$ er massene til klossene med tilhørende friksjonskoeffisienter $\mu_1$ og $\mu_2$.

\begin{figure}[h]
\begin{center}
\includegraphics[scale=0.3]{withforces}
\caption{Illustrasjon av oppsettet, der $F_n$ er normalkraften fra skråplanet på klossen, $F_r$ er friksjonskraften mellom klossen og underlaget, og $-mg$ er gravitasjonskraften.}
\end{center}
\label{oppsett}
\end{figure}

For å finne middelverdien til de utregnede friksjonskoeffisientene ble likningen
\begin{equation}
\overline{\mu} = \dfrac{1}{n}\sum_{i=1}^{n}\mu_i
\end{equation}
 der $n$ er antall målinger, brukt. Videre ble usikkerheten i middelverdien regnet ut ved følgende likning,
\begin{equation}
\delta\overline{\mu} = \dfrac{\delta\mu}{\sqrt{n}}.
\end{equation}

\chapter*{Fremgangsmåte}
\addcontentsline{toc}{chapter}{Fremgangsmåte}
\subsection*{Forberedelse}
\addcontentsline{toc}{subsection}{Forberedelse}
Et skjematisk oppsett av eksperimentet er vist i Figure 1. Treplanken ble festet til et stativ med en klype. Det ble brukt vater for å påse at skråplanets vinkel blir målt normalt til tyngekraften. Foran oppsettet ligger det en meterstokk, som brukes til å kalibrere størrelsesforhold i programvaren Tracker. Høyhastighetskameraet er montert normalt på oppsettet, slik at vinkling på kameraet ikke påvirker størrelsene som måles. To forskjellige underlag ble festet til henholdsvis en aluminium- og messingkloss. Figure 1 og Figure 2 viser de forskjellige underlagene.\\
\\Som foreberedelse til laben ble likningene i Teori delen utledet. Innstillingene på høyhastighetskameraet ble tilpasset lysforholdene, og satt til 100fps med automatisk hvitbalanse.\\
\\Ved hjelp av en vekt ble de to massenes vekt målt tre ganger for å verifiserte at målingene var representative tall for vekten til massene. En av vektene ble sluppet foran en hvit vegg mens kameraet filmet og deretter importert i Tracker. Ved hjelp av regresjonsfunksjonen i programmet ble gravitasjonsakselerasjonen funnet til å være i nærheten av 9.81 m/s$^2$, noe som verifiserte integriteten til utstyret. Deretter ble klossene sluppet ned langs skråplanet og friksjonskoeffisienten regnet ut og sammenliknet med teoretiske verdier funnet på internett [1].

\begin{figure}[h]
\begin{minipage}{.5\textwidth}
\centerline{\includegraphics[scale=0.05]{BlaaPlast}}
    \caption{Blå plast, friksjonskoeffisient: 0.270.}
    \label{fig:1}
\end{minipage}
\begin{minipage}{.5\textwidth}
    \centerline{\includegraphics[scale=0.05]{GultStoff}}
    \caption{Gul nylon, friksjonskoeffisient: 0.232.}
    \label{fig:2}

\end{minipage}
\end{figure}

\subsection*{Forsøk}
\addcontentsline{toc}{subsection}{Forsøk}
Forsøket ble gjennomført ved å først stille inn skråplanet til en vilkårlig vinkel. Med en linjal ble det tatt målinger av lengden og høyden av skråplanet som ble brukt til å bestemme vinkelen. Messingklossen ble sluppet 4 ganger nedover skråplanet på hvert underlag, og importert i Tracker for å tilpasse akselerasjon. Friksjonskoeffesienten ble kalkulert ved hjelp av likning (4). Deretter ble det gjenntatt med aluminiumsklossen for begge underlagene. \\
\\De kalkulerte friksjonskoeffesientene for underlagene ble brukt til å beregne den vinkelen hvor massesystemet sklei ned med konstant hastighet. Klossene ble bundet sammen med en snor, og sluppet ned skråplanet. Til slutt ble optimal vinkel kalkulert med ekstra vekter på 10 g og 50 g lagt på hver av klossene. Denne gangen ble det også målt kun en neglisjerbar akselerasjon.


%{\let\clearpage\relax\chapter*{Resultater og diskusjon}}
\chapter*{Resultater og diskusjon}
\addcontentsline{toc}{chapter}{Resultater og diskusjon}
\section*{Resultater}
\addcontentsline{toc}{section}{Resultater}
Nedenfor er resultatene som ble funnet ved å bruke Tracker til å finne akselerasjonen til klossene. Vinkelen ble også funnet i Tracker, ved å se på vinkel mellom skråplanet og vateret. Tabell 1 viser målingene og tilhørende utregnede friksjonskoeffisienter, Tabell 2 viser gjennomsnittlige verdier for disse målingene med usikkerhet. Videre viser Tabell 3 den utregnede vinkelen $\theta$ som skal gi konstant hastighet for massesystemet ned skråplanet. Tabell 4 viser akselerasjonen for klossene med utregnet $\theta$, som teoretisk sett gir null akselerasjon.
\begin{center}
\begin{tablenotes}
 	\small
 	\item Tabell 1, viser rådata for akselerasjon og utregnet friksjonskoeffisient for klosstypene på oppgitt underlag, der det ble gjennomført fire målinger for hver kombinasjon av klosstype og underlag. Merk at for måling 13 til 16 ble vinkelen økt for å få klossen til å slik. Ellers ble samme vinkel brukt i de resterende målingene, men usikkerheten ligger i oppsettet av målingene i Tracker.
 	\end{tablenotes}
  \begin{tabular}{| c | c | c | c | c | c |}
    \hline
    Måling & Klosstype & Underlag & Akselerasjon $[m/s^2]$ & Vinkel [$^{\circ}$] & Friksjonskoeffisient \\ \hline
    1 & Messing & Nylon & 0.248 & 17.8 & 0.268 \\ \hline
    2 & Messing & Nylon & 0.412 & 16.9 & 0.216 \\ \hline
    3 & Messing & Nylon & 0.370 & 16.8 & 0.223 \\ \hline
    4 & Messing & Nylon & 0.366 & 16.7 & 0.222 \\ \hline
    5 & Aluminium & Nylon & 0.339 & 17.2 & 0.237 \\ \hline
    6 & Aluminium & Nylon & 0.299 & 16.8 & 0.238 \\ \hline
    7 & Aluminium & Nylon & 0.294 & 17.8 & 0.258 \\ \hline
    8 & Aluminium & Nylon & 0.306 & 17.5 & 0.250 \\ \hline
    9 & Messing & Plast & 0.337 & 18.2 & 0.256 \\ \hline
    10 & Messing & Plast& 0.196 & 18.7 & 0.296 \\ \hline
    11 & Messing & Plast& 0.296 & 17.8 & 0.258 \\ \hline
    12 & Messing & Plast& 0.189 & 18.2 & 0.288 \\ \hline
    13 & Aluminium & Plast & 0.431 & 25.0 & 0.369 \\ \hline
    14 & Aluminium & Plast & 0.418 & 24.9 & 0.370 \\ \hline
    15 & Aluminium & Plast & 0.572 & 26.1 & 0.360 \\ \hline
    16 & Aluminium & Plast & 0.766 & 25.3 & 0.300 \\ \hline
  \end{tabular}
\end{center}

Ved å bruke likning (7) og (8) ble så rådata omregnet til gjennomsnittlige verdier med usikkerhet.

\begin{center}
 \begin{tablenotes}
 	\small
 	\item Tabell 2, viser snitt for akselerasjon, vinkel og friksjonskoeffisient for de forskjellige underlagne og klossene med usikkerhet.
 	\end{tablenotes}
  \begin{tabular}{| c | c | c | c | c |}
    \hline
    Klosstype & Underlag & Akselerasjon $\pm$ $\delta A$ $[m/s^2]$ & Vinkel $\pm$ $\delta V$ [$^{\circ}$] & Friksjonskoeffisient $\pm$ $\delta \mu$ \\ \hline
    Messing & Nylon & 0.349 $\pm$ $0.082$ & 17.1 $\pm$ $0.6$ & 0.232 $\pm$ $0.082$ \\ \hline
    Aluminium & Nylon & 0.310 $\pm$ $0.023$ & 17.3 $\pm$ $0.5$ & 0.246 $\pm$ $0.086$\\ \hline
    Messing & Plast & 0.255 $\pm$ $0.074$ & 18.2 $\pm$ $0.5$ & 0.270 $\pm$ $0.095$\\ \hline
    Aluminium & Plast & 0.547 $\pm$ $0.174$ & 25.3 $\pm$ $0.6$ & 0.350 $\pm$ $0.126$\\ \hline
  \end{tabular}
\end{center}

Verdiene for gjennomsnittlig friksjonskoeffisient ble så satt inn i likning (6), for å få verdier for vinkelen, $\theta$, som skulle gi konstant hastighet for det sammensatte systemet.

\begin{center}
     \begin{tablenotes}
 	\small
 	\item Tabell 3, viser vinkelen, $\theta$, som teoretisk sett skal gi konstant hastighet ned skråplanet, med usikkerhet.
 	\end{tablenotes}
  \begin{tabular}{| c | c | c |}
    \hline
    Stoff kloss 1 & Stoff kloss 2 & $\overline{\theta} \pm \delta\theta$ \\ \hline
    Nylon & Nylon & 13.6 $\pm$ 0.8 \\ \hline
    Plast & Plast & 16.4 $\pm$ 1.2 \\ \hline
    Nylon & Plast & 17.7 $\pm$ 1.7 \\ \hline
    Plast & Nylon & 14.7 $\pm$ 1.5  \\ \hline
  \end{tabular}
\end{center}
Tabellen nedenfor viser at dersom en tok den minste vinkelen som var innenfor usikkerheten, så ble akselerasjonen til systemet nesten lik null.

\begin{center}
     \begin{tablenotes}
 	\small
 	\item Tabell 4, viser resultatene fra målingene som ble gjennomført på massesystem med gitt vekt og vinkel for konstant hastighet, altså ingen akselerasjon. 
 	\end{tablenotes}
  \begin{tabular}{| c | c | c | c | c  | c |}
    \hline
    Underlag kloss 1 & Underlag kloss 2 & Vekt kloss 1 [kg] & Vekt kloss 2 [kg] & Vinkel $\theta$ & Akselerasjon $[m/s^2]$ \\ \hline
    Nylon & Nylon & 0.022 & 0.0687 & 12.8 & 0.12 \\ \hline
    Nylon & Plast & 0.0242 & 0.0687 & 16.0 & 0.02 \\ \hline
    Plast & Nylon & 0.0343 & 0.1197 & 13.3 & 0.04 \\ \hline
    Plast & Plast & 0.0343 & 0.1197 & 16.0 & lav, mistet data  \\ \hline
  \end{tabular}
\end{center}


\section*{Diskusjon og feilkilder}
\addcontentsline{toc}{section}{Diskusjon og feilkilder}
Som en ser i resultatene ovenfor, så ga alle de teoretisk utregnede verdiene en vinkel som var for stor, noe som førte til at klossene akselererte nedover skråplanet. En feilkilde som ble observert mens testene ble gjennomført var at klossene beveget seg litt hakkete nedover skråplanet. Dette er høyst sannsynlig forårsaket ved at skråplanet ikke har helt homogen overflate, noe som gjør at friksjonskoeffisientene endrer seg nedover langs skråplanet. Klossene ble ikke alltid sluppet på akuratt samme sted, da skråplanet var forholdsvis bredt, og overflaten kan også være forskjellig i bredden, noe som kan ha forårsaket feil i målingne. Ut ifra likning (4) ser en også at vinkelen, $\theta$ har mye å si, dette ble også observert ved de initielle målingene. At vateret ble brukt til å finne nøyaktig horisontal posisjon var svært viktig, men her kan det fortsatt være noe usikkerhet, med tanke på hvordan det ble overført til Tracker.\\\\
Måling 13 til og med 16 i Tabell 1, gir større verdier for friksjonskoeffisienten enn målingene 9 til 12 i samme tabell. Det ble også observert at klossen i måling 13 til 16 hadde høyere hastighet. Her er både klossen og vinkelen endret. Vel å merke her er at det er forskjell på friksjon avhengig av hastighet mellom de to underlagene. Det vil si at det for plasten var en annen friksjonskoeffisient ved høyere hastigheter enn ved lavere hastigheter. Dette er noe en kunne sett nærmere på, og utført flere målinger, for å sett om dette gjaldt for begge underlagene. Etter å ha sammenliknet friksjonskoeffisienten for metall og tre, som ble målt initielt, med verdier funnet på internett [1], kan en konkludere med at målingene for friksjonskoeffisientene ligger innenfor det område som kan forventes.\\\\
En annen ting som ble observert i målingene som ble gjennomført helt i starten av en kloss i fritt fall, var at selv om klossen falt fritt, så stemte ikke alltid akselerasjonsverdiene som ble funnet i Tracker. Verdiene for gravitasjonsakselerasjonen lå i nærheten av det en skulle forvente, men at verdiene hadde en usikkerhet på rundt $\pm$ 20$\%$. Derfor ble det tatt totalt 8 målinger av hver stofftype, men for enda mer presise resultater kunne det blitt gjort en rekke flere målinger.

{\let\clearpage\relax\chapter*{Konklusjon}}
\addcontentsline{toc}{chapter}{Konklusjon}
Hypotesen vår var at det var mulig å finne en optimal vinkel for konstant hastighet for et massesystem med gitt underlag, noe som kan konkluderes med at ble oppnådd, da hastigheten var nesten konstant, med akselerasjoner på mellom 0.12 $[m/s^2]$ til 0.02 $[m/s^2]$. Det var en del feilkilder som gjorde at resultatene ikke ble slik som forventet, dette gjelder blandt annet uhomogen overflate på skråplanet, regresjon i Tracker og unøyaktigheter i oppsett av måling i Tracker. Det opplevdes ofte at den kalkulerte vinkelen var for høy.

\chapter*{Referanser}
[1] \href{url}{https://en.wikipedia.org/wiki/Friction}
Nedlastingsdato: 26. oktober 2016

\end{document}

