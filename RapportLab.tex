\documentclass[10pt,a4paper]{report}
\usepackage[utf8]{inputenc}
\usepackage{amsmath}
\usepackage{amsfonts}
\usepackage{amssymb}
\usepackage{graphicx}
\usepackage[margin=1in]{geometry}
\author{Finn Matras}
\begin{document}
\chapter*{Forord}
Hvem har skrevet hva???????SDF??SD?

\chapter*{Innhold}
\begin{itemize}
\item Sammendrag
\item Innledning
\item Teori
\item Eksperimentelt
\item Resultater og diskusjon
\item Konklusjon
\end{itemize}

\chapter*{Sammedrag}
I dette forsøket var formålet å finne friksjon mellom to ulike stoffer og et skråplan, for at man skal kunne regne seg fram til en den horisontale vinkelen til skråplanet som gjør at et system bestående av to masser på et gitt underlag sklir med konstant hanstighet ned skråplanet.\\
\\For å finne friksjonskoeffisienten bruke vi videoanalyse-programvaren Tracker og et høyhastighetskamera for å analysere bevegelsen til klossene nedover langs skråplanet. I Tracker kunne vi lese av akselerasjonen til klossen, og ved hjelp av Newtons II lov kunne vi regne oss frem til friksjonskoeffisienten til underlagene. Målingene ble gjenntatt flere ganger for å verifisere og ta høyde for usikkerhet.\\
\\Vi fant ut at friksjonskoeffisienten mellom sykkelsete-stoff (Figur: 1) og skråplanet i tre var: s234234234234, og at friksjonskoeffisienten mellom plast og tre (Figur: 2) var:234234


\chapter*{Innledning}
\begin{figure}[p]
    \includegraphics[scale=0.05]{BlaaPlast}
    \caption{Blå plast, friksjonskoeffisient: 234324324}
    \label{fig:1}
\end{figure}
\begin{figure}[p]
    \includegraphics[scale=0.05]{GultStoff}
    \caption{Gult plaststoff, friksjonskoeffisient: 234324324}
    \label{fig:2}
\end{figure}
\section*{Teori}
Basert på Newtons II low (2) 
\begin{equation}
\sum{F} = m*a
\end{equation} så kan vi regne ut kreftene som er tilstede på en masse. 
\begin{equation}
F = \mu *F_n
\end{equation}
Likning (3) brukes til å regne ut friksjonskoeffisienten, $\mu$. Ved å regne på likningene over, så får vi likning (1) og (4)
\begin{equation}
\mu = tan(\theta)-a/(cos(\theta)*g)
\end{equation}
Vi kan så bruke denne liknignen til å regne ut friksjonskoeffisienten for de forskjellige akselerasjonsmålingene. Se figur 324532432423. 
\begin{equation}
\theta = arctan(\frac{\mu_1*m_1+\mu_2*m_2}{m_1+m_2})
\end{equation}
Ved hjelp av likning (1) kan man så regne seg fram til vinkelen, $\theta$, som gjør at klossene med masser $m_1$ og $m_2$ sklir med konstant fart ned langs skråplanet for et massesystem på et gitt underlag med friksjonskoeffisient, $\mu$.\\
\\I dette forsøket var formålet å bli kjent med hvordan man utfører forsøk i fysikken, og å bli kjent med hvordan man skriver lab-journal og rapport. Vi har lært mye om usikkerhet og hvordan feil i det tekniske utstyret kan føre til følgefeil. Videre har vi også lært oss hvordan vi bør føre journal for at vi eller andre i etterkant selv skal kunne gjennomføre forsøket på nytt.

\chapter*{Utstyr og metoder}
\section*{Utstyr}
\section*{Metode}
\subsection*{Forberedelse}
\subsection*{Forsøk}
\chapter*{Resultater og diskusjon}
\section*{Resultater}
Nedenfor er resultatene vi fikk ved å bruke Tracker til å finne akselerasjonen til klossene, og usikkerheten til disse målingene.
\begin{center}
  \begin{tabular}{| c | c | c | c | c | c |}
    \hline
    Måling & Klosstype & Materiale & Akselerasjon $[m/s^2]$ & Vinkel [$\deg$] & Fiksjonskoeffisient | \\ \hline
    1 & Gull & Sykkelsetestoff & 0,248 & 17,8 & 0,268 \\ \hline
    2 & Gull & Sykkelsetestoff & 0,412 & 16,9 & 0,216 \\ \hline
    3 & Gull & Sykkelsetestoff & 0,37 & 16,8 & 0,223 \\ \hline
    4 & Gull & Sykkelsetestoff & 0,366 & 16,7 & 0,222 \\ \hline
    Snitt & Gull & Sykkelsetestoff & 0,349 & 17,1 & 0,232 \\ \hline
    Avvik & Gull & Sykkelsetestoff & 0,082 & 0,55 & 0,0260 \\
    \hline
  \end{tabular}
\end{center}


\begin{center}
  \begin{tabular}{| c | c | c | c | c | c |}
    \hline
    Måling & Klosstype & Materiale & Akselerasjon $[m/s^2]$ & Vinkel [$\deg$] & Fiksjonskoeffisient | \\ \hline
    5 & Aluminium & Sykkelsetestoff & 0,339 & 17,2 & 0,237 \\ \hline
    6 & Aluminium & Sykkelsetestoff & 0,299 & 16,8 & 0,238 \\ \hline
    7 & Aluminium & Sykkelsetestoff & 0,294 & 17,8 & 0,258 \\ \hline
    8 & Aluminium & Sykkelsetestoff & 0,306 & 17,5 & 0,250 \\ \hline
    Snitt & Aluminium & Sykkelsetestoff & 0,310 & 17,3 & 0,246 \\ \hline
    Avvik & Aluminium & Sykkelsetestoff & 0,023 & 0,50 & 0,011 \\
    \hline
  \end{tabular}
\end{center}


\begin{center}
  \begin{tabular}{| c | c | c | c | c | c |}
    \hline
    Måling & Klosstype & Materiale & Akselerasjon $[m/s^2]$ & Vinkel [$\deg$] & Fiksjonskoeffisient | \\ \hline
    9 & Gull & Plastik & 0,337 & 18,2 & 0,256 \\ \hline
    10 & Gull & Plastik& 0,196 & 18,7 & 0,296 \\ \hline
    11 & Gull & Plastik& 0,296 & 17,8 & 0,258 \\ \hline
    12 & Gull & Plastik& 0,189 & 18,2 & 0,288 \\ \hline
    Snitt & Gull & Plastik & 0,255 & 18,2 & 0,27 \\ \hline
    Avvik & Gull & Plastik & 0,074 & 0,45 & 0,020 \\
    \hline
  \end{tabular}
\end{center}

For å beregne farten til aluminiumsklossen med plastunderlag, så måtte vi endre vinkelen for at klossen skulle skli.
\begin{center}
  \begin{tabular}{| c | c | c | c | c | c |}
    \hline
    Måling & Klosstype & Materiale & Akselerasjon $[m/s^2]$ & Vinkel [$\deg$] & Fiksjonskoeffisient | \\ \hline
    13 & Aluminium & Plastik & 0,431 & 25 & 0,369 \\ \hline
    14 & Aluminium & Plastik & 0,418 & 24,9 & 0,370 \\ \hline
    15 & Aluminium & Plastik & 0,572 & 26,1 & 0,360 \\ \hline
    16 & Aluminium & Plastik & 0,766 & 25,3 & 0,300 \\ \hline
    Snitt & Aluminium & Plastik & 0,547 & 25,3 & 0,350 \\ \hline
    Avvik & Aluminium & Plastik & 0,174 & 0,60 & 0,035 \\
    \hline
  \end{tabular}
\end{center}

Ved å sette verdiene for snitt, med usikkerhet inn i linkning (1), og få følgende verdier for vinkelen, $\theta$, for et system bestående av to klosser som i illustrasjon 32234532423432:

\begin{center}
  \begin{tabular}{| c | c | c | c | c |}
    \hline
    Stoff kloss 1 & Stoff kloss 2 & $\theta_snitt$ & $\theta_maks$ & $\theta_min$ | \\ \hline
    Sykkelsetestoff & Sykkelsetestoff & 13,6 & 14,4 & 12,8 \\ \hline
    Plast & Plast & 16,4  & 17,6 & 15,1 \\ \hline
    Sykkelsetestoff & Plast & 17,7 & 19,4 & 15,2 \\ \hline
    Plast & Sykkelsetestoff & 14,7 & 16,2 & 13,2  \\ \hline
  \end{tabular}
\end{center}

Testing av utregnede verdier på et massesystem som vist i illustrason 32432423 ga oss følgende akselerasjoner:

\begin{center}
  \begin{tabular}{| c | c | c | c | c  | c |}
    \hline
    Test nr. & Stoff & Vekt kloss 1 & Vekt kloss 2 & Vinkel $\theta$ & Akselerasjon $[m/s^2]$ | \\ \hline
    1 & Sykkelsetestoff & 0,022 & 0,0687 & 12,8 & 0,12 \\ \hline
    2 & Sykkelsetestoff & 0,0242  & 0,0687 & 12,4 & 0,0216 \\ \hline
    3 & Sykkelsetestoff & 0,0343 & 0,1197 & 12,9 & 0,04 \\ \hline
    4 & Plast & 0,0343 & 0,1197 & 16 & lav  \\ \hline
  \end{tabular}
\end{center}

\section*{Diskusjon og feilkilder}
Som vi ser i resultatene ovenfor, så ga alle de teoretisk utregnede verdiene en vinkel som var for stor, noe som førte til at klossene akselererte nedover skråplanet. En feilkilde som ble observert mens vi gjennomførte disse testene som hadde ensten konstant hastighet var at klossene beveget seg litt hakkete nedover skråplanet. Dette er mest sannsynlig forårsaket ved at skråplanet ikke har helt monoton overflate, noe som gjør at klossene og friksjonskoeffisientene endrer seg underveis. Vi klarte heller ikke å slippe klossene ned på akuratt samme sted, da skråplanet var forholdsvis bredt, og overflaten kan også være forskjellig i bredden, noe som kan ha forårsaket feil i målingne. Ut ifra likningen ser man også at vinkelen, $\theta$ har mye å si, og dette merket vi også i starten da vi gjorde noen initielle målinger. Det at vi bruke et vater til å finne nøyaktig horisontal posisjon var svært viktig, men her kan det fortsatt være en del usikkerhet, med tanke på hvordan det ble overført til Tracker. \\
\\På måling 13 til 16 ser vi også at friksjonskoeffisienten er en del større, enn den er på måling 9 til 12. Her har vi både endret massen, og vinkelen. Vel å merke her er at det er forskjell på friksjon avhengig av forskjellen i hastighet mellom de to underlagene. Det vil si at det for plast-stoffet var en annen friksjonskoeffisient ved høyere hastigheter enn ved lavere hastigheter. Det vil si at det kunne vært lurt å prøve seg fram til den vinkelen der det er konstant hastighet eksperimentelt, og så analysere akselerasjonen ved den funnede vinkelen for å finne friksjonskoeffisienten slik den er i nærheten av den hastigheten som den kommer til å bli utsatt for.\\
\\En annen ting vi merket vålingene som vi gjorde helt i starten av en kloss i fritt fall, var at selv om klossen falt fritt, så stemte ikke alltid akselerasjonsverdiene som vi fikk ved regresjonen i Tracker. Det vi så var at disse verdiene lå i nærheten av det vi skulle forvente, men at verdiene hadde en usikkerhet på rundt $\pm 20 prosent$. Derfor valgte vi å ta totalt 8 målinger av hver stofftype, men for enda mer presise resultater kunne det blitt gjort en rekke flere målinger.
\chapter*{Konklusjon}
\end{document}